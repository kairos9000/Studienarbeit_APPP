\chapter{Vorbereitung und Vorgehensweise}
\label{chap:2}

Bevor die App entwickelt werden konnte waren einige Vorbereitungen nötig. Zuerst wurde eine Umgebung benötigt, in der die Entwicklung stattfinden konnte. Dafür wurde zuerst ein Template installiert, welches von Expo bereitgestellt wurde\cite{expo}. Expo ist ein weiteres Framework, welches die Arbeit mit react-native erleichtert. Durch Expo lassen sich Pakete für die Apps installieren, Entwicklungsserver starten und vieles mehr. Es wurde sich hier für ein Template mit TypeScript entschieden, da es sich dabei um typsicheres JavaScript handelt, wobei viele Fehler schon in der Entwicklung gefunden werden können.

Um die App testen zu können wurde zudem ein Smartphone benötigt. Ein reales Smartphone ist zwar möglich, jedoch kann hier nicht die Position des Nutzers nach Belieben verändert werden, was sehr unvorteilhaft für die Entwicklung einer App ist, welche bestimmte Ereignisse nur an bestimmten Positionen ermöglicht. Also wurde hier über Android Studio ein Android-Emulator installiert. Es wurde sich für einen Emulator des Pixel 3a XL Smartphones entschieden, da hier ein ausreichend großer Bildschirm vorhanden ist.

Ein weiterer Vorbereitungsschritt bestand darin, die statischen Daten der Parkhäuser zu finden. Durch die API der Stadt Amberg können nur Daten abgefragt werden, die sich ändern, also dynamische Daten, wie freie Parkplätze oder der Trend mit dem sich das Parkhaus füllt oder leert. Hier fehlen jedoch Daten wie die Koordinaten des Parkhauses oder die stündlichen Preise für das Parken diese Daten wurden zuerst unter der URL \url{https://www.amberg.de/leben-in-amberg/mobilitaet/parken} bereitgestellt, während der Entwicklung der App wurden die Daten jedoch auf die Webseite \url{https://www.amberg.de/parken} verschoben. Hier konnten die Koordinaten der einzelnen Parkhäuser über eine Webseiteninterne OpenStreetMap-Karte ausgelesen werden, wobei die Koordinaten für die spätere Navigation ein wenig angepasst wurden, sodass diese genau auf den Eingängen zu den einzelnen Parkhäusern liegen. Die Preise und Öffnungszeiten wurden dann durch die weiteren Informationen auf der Webseite und eine von der Stadt Amberg bereitgestellte umfassende Übersicht über die Preise als PDF-Datei zusammengestellt. Die PDF-Datei ist unter dem Link \url{https://www.amberg.de/fileadmin/Mobilitaet/Parkpreise_Uebersicht.pdf} verfügbar. Zudem wurden noch nützliche Zusatzinformationen übernommen, wie zum Beispiel, dass das Parken für Gäste des Kurfürstenbades auf dem Parkplatz des Kurfürstenbades und dem Parkhaus Kurfürstengarage frei ist. Alle diese Daten sind in der Datei \verb|staticDataParkingGarage.ts| als JavaScript-Objekt für die einfachere Handhabung zu finden.

Nachdem alle Daten gesammelt waren, mussten diese noch persistent gespeichert werden. Hier gab es zwei Möglichkeiten: Die erste ist die Speicherung in einer Datenbank, hier ist SQLite unter react-native verfügbar. Die zweite und modernere Variante ist das Paket async-storage. Zuerst wurde versucht eine Datenbank aufzusetzen über das expo-sqlite Paket, welches von wieder Expo bereitgestellt wird \cite{expo-sqlite}. Hier mussten SQL-Abfragen benutzt werden, um die Daten zu speichern und auszulesen. Da dies zu Unmengen an Code führte, wurde die zweite Möglichkeit versucht \cite{async-storage}. Das async-storage Paket speichert Daten als Schlüssel-Wert Paare, das heißt die Daten bekommen bei der Speicherung einen Schlüssel als String zugewiesen, mit dem sie danach identifiziert und wieder ausgelesen werden können. Die Daten müssen dabei auch als String gespeichert werden. Der Vorteil bei dieser Methode ist, dass JavaScript-Objekte einfach in einen String konvertiert und beim Auslesen wieder in ein JavaScript-Objekt rekonvertiert werden können. Auch sind keine Datenbankstrukturen nötig, wie Tabellen und Beziehungen zwischen diesen.

In \autoref{lst:staticData} ist das Format der statischen Daten am Beispiel des Parkhauses am Ziegeltor zu sehen.

  \begin{lstlisting}[language=JavaScript, caption={Format der statischen Daten der Parkhäuser am Beispiel des Parkhauses am Ziegeltor}, label=lst:staticData]
id: 4,
name: "Am Ziegeltor",
coords: {
	latitude: 49.44864,
	longitude: 11.85684,
},
numberOfParkingSpots: 200,
pricingDay: {
	firstHour: 1,
	followingHours: 0.5,
	maxPrice: 5,
	startHours: "08:00",
	endHour: "19:00",
},
pricingNight: {
	firstHour: 0.5,
	followingHours: 0.5,
	maxPrice: 1.5,
	startHours: "19:00",
	endHour: "08:00",
},
openingHours: {
	startHour: "00:00",
	endHour: "24:00",
},
additionalInformation: "FLEXI-Ticket moeglich",
favorite: false,
	
\end{lstlisting}

In \autoref{lst:dynamicData} dagegen das Format der dynamischen Daten desselben Parkhauses.

  \begin{lstlisting}[language=JavaScript, caption={Format der dynamischen Daten der Parkhäuser am Beispiel des Parkhauses am Ziegeltor}, label=lst:dynamicData]
"Aktuell": 36,
"Frei": 164,
"Gesamt": 200,
"Geschlossen": 0,
"ID": 4,
"Name": "Am Ziegeltor",
"Status": "OK",
"Trend": -1
	
\end{lstlisting}

Nachdem diese Daten nun gespeichert werden konnten, war ein kontinuierliches Abfragen der API nötig, um immer die aktuellsten Daten zu besitzen. Hierfür wurde eine Funktionskomponente in react-native erstellt, welche im Ordner \verb|ParkingAPI| in der Datei \verb|useAPICall.ts| zu finden ist und bei Aufruf ein Intervall erzeugt. Dieses Intervall ruft nach einer bestimmten Zeit eine Funktion auf, welche die Daten der API anfragt. Um nicht zu viele Anfragen zu tätigen, aber trotzdem aktuelle Daten zu besitzen, ruft das Intervall die Funktion alle 60 Sekunden auf. Die Funktion zur Anfrage der API achtet zudem auch darauf, ob eine Internetverbindung besteht oder nicht. Falls keine Internetverbindung besteht, wird automatisch durch die fetch-Funktion, welche benutzt wird, um die API abzufragen, ein Fehler geworfen. Bei einem solchen Fehler werden die alten Daten, die noch im async-storage bestehen, verwendet. Falls keine solchen Daten bestehen, wird der Nutzer benachrichtigt, seine Internetverbindung zu prüfen. Falls jedoch die Daten aus der API abgefragt werden können, überschreiben die neuen Daten die alten im async-storage und die neuesten Daten stehen der App damit zur Verfügung. Über jedes dieser Ereignisse wird der Nutzer zudem über Benachrichtigungen informiert.

Da die Vorbereitung damit abgeschlossen war, konnte nun begonnen werden die App zu entwickeln. Hier stellte sich die Frage, welche Funktionalität zuerst entwickelt werden sollte. Es wurde sich zuerst für die Karte entschieden, da diese zum größten Teil durch Einbinden eines Expo-Pakets implementiert werden konnte. Für die Karte wurden zudem auch gleich die Konfigurationsmöglichkeiten umgesetzt. Danach wurde eine Liste entwickelt, mit welcher der Nutzer die Parkhäuser übersichtlich und mit Detail-Informationen betrachten konnte. Hier wurden auch wieder Konfigurationsmöglichkeiten implementiert. Die Pseudo-Navigation wurde als nächstes umgesetzt. Zuletzt wurden die zusätzlichen sinnvollen Funktionen entwickelt, welche die App benutzerfreundlicher machen sollten.
%
%\begin{itemize}
%
%	\item Welche Vorbereitung war nötig? => Expo-App erstellen mit TypeScript für mehr Typsicherheit; Erstellen eines Android Emulators über Android Studio, der dann die App anzeigt und dafür sorgt, dass der Standort beliebig gesetzt werden kann; Daten aus Webseite rausholen und aus PDF Datei, die jetzt nicht mehr verfügbar ist
%	\item Koordinaten der Parkhäuser über Webseite mit OpenStreetMap herausfinden => speziell die Eingänge zu den Parkhäusern für bessere Navigation
%	\item Aufsetzen einer sqlite-Datenbank => war sehr umständlich und nicht praktikabel => Umstieg auf async-storage
%	\item Format der Daten, wie sie in async-storage gespeichert sind zeigen => aus Konsole kopieren und von VS-Code in JSON formatieren lassen
%		\item Nach welcher Reihenfolge sollten die einzelnen Dinge (Karte, Liste, Navigation, extra Funktionen) implementiert werden?
%\end{itemize}