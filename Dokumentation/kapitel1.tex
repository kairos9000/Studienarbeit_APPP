\chapter{Problemstellung}
\label{chap:1}

Ziel dieser Studienarbeit war, das vorhandene Parkleitsystem Ambergs zu nutzen, um eine App zu entwickeln, welche die Daten der Parkhäuser übersichtlich und benutzerfreundlich darstellt. Hierbei wurden die Daten als XML-Datei live eine API der Stadt Amberg zur Verfügung gestellt. Über \url{https://parken.amberg.de/wp-content/uploads/pls/pls.xml} können diese abgefragt werden. Die App soll es dem Nutzer ermöglichen alle Daten der Parkhäuser einsehen zu können, um eine fundierte Entscheidung zu treffen, wo er parken will. Zur Entscheidungsfindung werden von der Stadt Amberg Daten für freie Parkplätze, die stündlichen Kosten des Parkens, der Trend, also ob sich das Parkhaus gerade füllt oder leert, und weiteres bereitgestellt. Zudem soll eine simple Navigation möglich sein, um den Nutzer nach seiner Entscheidung zu dem gewünschten Parkhaus zu leiten.

Zur Entwicklung der App wurde das Framework react-native verwendet, womit native Apps entwickelt werden können \cite{ReactNative}. Dies bedeutet, dass die Apps speziell für eine bestimmte Plattform, wie Android oder iOS, entwickelt werden. Der Vorteil bei react-native ist, dass hierfür jedoch nicht für jede Plattform eigener Code geschrieben werden muss, sondern Webtechnologien verwendet werden können und sich react-native unter der Haube um die plattformspezifische Umsetzung kümmert.

Im Folgenden wird beschrieben, wie die App aufgebaut ist und wie die einzelnen Funktionen umgesetzt wurden. Zudem werden auch zusätzliche Funktionen vorgestellt, welche bei der Entwicklung als sinnvoll erschienen und deshalb für eine bessere Benutzerfreundlichkeit auch implementiert wurden.

