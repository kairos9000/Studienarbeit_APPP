% LaTeX-Vorlage zur Erstellung einer Abschlussarbeit in der Fakultät Elektrotechnik, Medien und Informatik an der OTH Amberg-Weiden
% Diese Vorlage entstand im Rahmen des Kurses "LaTeX fürs Studium"
% Aktuelle Version: v0.02
% Stand: 06.08.2015
%
% Changelog:
%
% v0.02: 06.08.2015, Anpassung der Vorlage:
% + Persönliche Informationen (Vorname, Name, Titel usw.) werden direkt in die PDF-Dokumenteinstellungen übernommen
% + Korrektur der Verlinkung von Abbildungs- und Tabellenverzeichnis aus dem Inhaltsverzeichnis (phantomsection) bzw. deren Seitenzahl
%   Besten Dank für diesen Hinweis an Jan-Olaf Becker
% + Anpassung des Namens der Fakultät nach deren Umbenennung
%
% v0.01: 14.03.2012, Erstellung der Vorlage

\documentclass[12pt,oneside]{report}
\usepackage[T1]{fontenc}		% Einstellungen fuer Umlaute usw.
\usepackage[utf8x]{inputenc}
\usepackage[ngerman]{babel}

\usepackage{parskip}			% Einstellungen fuer Absaetze: Abstand statt Einrueckung

\usepackage[a4paper,			% Papierformat A4
	    left=2.5cm,				% linker Rand
	    right=2.5cm,			% rechter Rand
	    top=1.5cm,				% oberer Rand
	    bottom=1.5cm,			% unter Rand
	    marginparsep=5mm,		% Abstand der Randnotizen
	    marginparwidth=10mm, 	% Breite der Randnotizen
	    headheight=7mm,			% Hoehe der Kopfzeile
	    headsep=1.2cm,			% Abstand der Kopfzeile
	    footskip=1.5cm,			% Abstand der Fusszeile
	    includeheadfoot]{geometry}

\usepackage{fancyhdr}						% Konfiguration von Kopf- und Fusszeilen
\pagestyle{fancy}							% Seitenstil 'fancy'
\fancyhf{}									% vorhandene Einstellungen loeschen
\setlength{\headwidth}{\textwidth}			% Kopf- und Fusszeile so breit wie der Haupttext
\fancyfoot[R]{\thepage} 					% Festlegung des Seitenstils: Seitenzahlen in der Fusszeile rechts
\fancyfoot[L]{\leftmark}					% Kapitelnr. und -Bezeichnung in der Fusszeile links
\fancyhead[R]{\IhreArbeit}					% "Bachelorarbeit" in der Kopfzeile rechts
\fancyhead[L]{\IhrVorname\ \IhrNachname}	% Vorname und Name in der Kopfzeile links
\renewcommand{\chaptermark}[1]{			% Definition der Ausgabe des Kapitels
  \markboth{Kapitel \thechapter. #1}{}}
\renewcommand{\headrulewidth}{0.5pt}		% Trennlinie zwischen Kopfzeile und Haupttext
\renewcommand{\footrulewidth}{0.5pt}		% Trennlinie zwischen Haupttext und Fusszeile
\fancypagestyle{plain}{					% Anpassung des Seitenstils 'plain' bei Beginn neuer Kapitel
  \fancyhf{}								% Vorbelegung loeschen
  \fancyfoot[C]{\thepage}					% Seitenzeilen in der Fusszeile mittig
  \fancyhead[R]{\IhreArbeit}				% "Bachelorarbeit" in der Kopfzeile rechts
  \fancyhead[L]{\IhrVorname\ \IhrNachname}	% Vorname und Name in der Kopfzeile links
}

\usepackage{amsmath}			% Pakete fuer den Mathematikmodus
\usepackage{amssymb}
\usepackage[intlimits]{empheq}

\usepackage[sc]{mathpazo}		% Schriftart Palatino fuer Haupttext und Mathematikmodus
\usepackage{pifont}				% zusaetzliche Symbole

\usepackage[format=hang,		% Einstellung fuer Bildunterschriften
            font={footnotesize},
            labelfont={bf},
            margin=1cm,
            aboveskip=5pt,
            position=bottom]{caption}

\usepackage{graphicx}							% Einbinden von Graphiken
\usepackage[svgnames,cmyk,table,hyperref]{xcolor} 	% Verwendung von Farben
\usepackage{tikz}								% Erstellen von Grafiken
\usetikzlibrary{positioning,arrows,plotmarks} % TikZ-Bibliotheken
%\usepackage{pgfplots}                           % Darstellung von Plots, Funktionen, Graphen usw.

%
% Weitere Pakete
%
%\usepackage{listings}			% Darstellung von Quellcode
%\lstset{language=Python, basicstyle=\ttfamily, numbers=none}
%
%\usepackage[european, siunitx]{circuitikz}	% Darstellung von Schaltungen
%
%\usepackage{enumerate}			% Formatierung nummerierter Listen
\usepackage{wrapfig}
\usepackage{listings}
\definecolor{purple}{rgb}{0.65, 0.12, 0.82}
\definecolor{commentGreen}{rgb}{0.0, 0.5, 0.0}
\lstdefinelanguage{JavaScript}{
	keywords=[1]{false, null, true, boolean, number, undefined, const, break, continue, delete, else, for, function, if, in,
		new, return, this, typeof, var, let, void, while, with},
	keywordstyle=\color{blue}\bfseries,
	% Literals, primitive types, and reference types.
	ndkeywords=[2]{export, import,
		Array, Boolean, Date, Math, Number, String, Object, await},
	ndkeywordstyle=\color{purple}\bfseries,
	% Built-ins.
	morekeywords=[3]{eval, parseInt, parseFloat, escape, unescape},
	sensitive,
	morecomment=[s]{/*}{*/},
	morecomment=[l]//,
	morecomment=[s]{/**}{*/}, % JavaDoc style comments
	commentstyle=\color{commentGreen}\ttfamily,
	stringstyle=\color{red}\ttfamily,
	morestring=[b]',
	morestring=[b]"
}[keywords, comments, strings]

\lstset{
	extendedchars=true,
	basicstyle=\footnotesize\ttfamily,
	showstringspaces=false,
	showspaces=false,
	numbers=left,
	numberstyle=\footnotesize,
	numbersep=9pt,
	tabsize=2,
	breaklines=true,
	showtabs=false,
	captionpos=b,
	xleftmargin=4em
}
\usepackage{microtype,relsize}					% Wird verwendet, um Nachnamen auf Titelseite gesperrt darzustellen
\newcommand*{\Sperren}[1]{\textls*[100]{#1}}

% 
% Persoenliche Angaben
% 
\newcommand*{\IhrVorname}{Philip}
\newcommand*{\IhrNachname}{Bartmann}
\newcommand*{\IhrStudiengang}{Medieninformatik}
\newcommand*{\IhreArbeit}{App-Programmierung Studienarbeit\\ Wintersemester 2022/23}
\newcommand*{\IhrTitelDE}{Dokumentation der Entwicklung und Implementierung einer App zur Visualisierung von Daten der Parkhäuser Ambergs mithilfe des Frameworks react-native}
\newcommand*{\IhrTitelEN}{\LaTeX\ under control}
\newcommand*{\IhrBearbeitungszeitraumVON}{22. November 2022}
\newcommand*{\IhrBearbeitungszeitraumBIS}{17. Januar 2023}
\newcommand*{\IhrErstpruefer}{Prof. Dr. Axel Maier}
\newcommand*{\IhrZweitpruefer}{Prof. Dr. Bruno Müller}
\newcommand*{\IhreFirma}{Renner-Verlag, Musterstadt}
\newcommand*{\IhrFirmenbetreuer}{Dr. Karl Schulze}
\newcommand*{\IhreZusammenfassung}{%
Lorem ipsum dolor sit amet, consetetur sadipscing elitr, sed diam nonumy eirmod tempor invidunt ut labore et dolore magna aliquyam erat, sed diam voluptua. At vero eos et accusam et justo duo dolores et ea rebum. Stet clita kasd gubergren, no sea takimata sanctus est Lorem ipsum dolor sit amet. Lorem ipsum dolor sit amet, consetetur sadipscing elitr, sed diam nonumy eirmod tempor invidunt ut labore et dolore magna aliquyam erat, sed diam voluptua. At vero eos et accusam et justo duo dolores et ea rebum. Stet clita kasd gubergren, no sea takimata sanctus est Lorem ipsum dolor sit amet.
}
\newcommand*{\IhreSchluesselwoerter}{Textsatz, Formeln, Graphiken}


\usepackage[bookmarks, raiselinks, pageanchor, % PDF-Einstellungen
            hyperindex, colorlinks,
            citecolor=black, linkcolor=black,
            urlcolor=black, filecolor=black,
            menucolor=black]{hyperref}
\hypersetup{pdftitle={\IhrTitelDE},%
            pdfauthor={\IhrVorname\ \IhrNachname},%
            pdfsubject={\IhreArbeit},%
            pdfkeywords={\IhreSchluesselwoerter}}

%
% Beginn des Textteils
%
\begin{document}
  \pagenumbering{roman}
  \begin{titlepage}	
  						% Titelseite
    \thispagestyle{empty}
    \begin{center}
    	\let\clearpage\relax
    	\include{formblatt_selbststaendigkeitserklaerung}

      \Large
      Ostbayerische Technische Hochschule Amberg-Weiden\\
      Fakultät Elektrotechnik, Medien und Informatik\\[1cm]
      Studiengang \IhrStudiengang\\[1cm]
      \textbf{\IhreArbeit}\\[1cm]
      von\\[1cm]
      \IhrVorname\ \Sperren{\textbf{\IhrNachname}}\\[1cm]
      \textbf{\IhrTitelDE}\\[1cm]
%      \IhrTitelEN
    \end{center}
  \end{titlepage}
%  \clearpage
%  \thispagestyle{empty}			% 1. Seite soll eine Leerseite sein (dazu muss ein Trick verwendet werden)
%  \mbox{}
%  \clearpage
%  \thispagestyle{empty}			% 2. Seite wie Titelseite, aber mit zusaetzlichen Angaben
%  \begin{center}
%    \Large
%    Ostbayerische Technische Hochschule Amberg-Weiden\\
%    Fakultät Elektrotechnik, Medien und Informatik\\[1cm]
%    Studiengang \IhrStudiengang\\[1cm]
%    \textbf{\IhreArbeit}\\[1cm]
%    von\\[1cm]
%    \IhrVorname\ \Sperren{\textbf{\IhrNachname}}\\[1cm]
%    \textbf{\IhrTitelDE}\\[1cm]
%    \IhrTitelEN
%  \end{center}
%  \vspace*{5cm}
%  \begin{tabbing}
%    \underbar{Bearbeitungszeitraum:}\qquad\= von\qquad\=\IhrBearbeitungszeitraumVON\\
%                                          \> bis      \>\IhrBearbeitungszeitraumBIS
%  \end{tabbing}
%  \vspace*{1cm}
%  \underbar{1. Prüfer:}\qquad\IhrErstpruefer\par 
%  \underbar{2. Prüfer:}\qquad\IhrZweitpruefer
%  \clearpage
  	% 3. Seite: Formblatt Bestaetigung nach Paragraph 12 APO
%  \include{formblatt_summary}			% 4. Seite: Formblatt Zusammenfassung
  \tableofcontents
%  \newpage
%  \chapter*{Symbole, Formelzeichen und Einheiten}
  \newpage
  \pagenumbering{arabic}
  \chapter{Problemstellung}
Lorem ipsum dolor sit amet, consetetur sadipscing elitr, sed diam nonumy eirmod tempor invidunt ut labore et dolore magna aliquyam erat, sed diam voluptua. At vero eos et accusam et justo duo dolores et ea rebum. Stet clita kasd gubergren, no sea takimata sanctus est Lorem ipsum dolor sit amet. Lorem ipsum dolor sit amet, consetetur sadipscing elitr, sed diam nonumy eirmod tempor invidunt ut labore et dolore magna aliquyam erat, sed diam voluptua. At vero eos et accusam et justo duo dolores et ea rebum. Stet clita kasd gubergren, no sea takimata sanctus est Lorem ipsum dolor sit amet.

Lorem ipsum dolor sit amet, consetetur sadipscing elitr, sed diam nonumy eirmod tempor invidunt ut labore et dolore magna aliquyam erat, sed diam voluptua. At vero eos et accusam et justo duo dolores et ea rebum. Stet clita kasd gubergren, no sea takimata sanctus est Lorem ipsum dolor sit amet. Lorem ipsum dolor sit amet, consetetur sadipscing elitr, sed diam nonumy eirmod tempor invidunt ut labore et dolore magna aliquyam erat, sed diam voluptua. At vero eos et accusam et justo duo dolores et ea rebum. Stet clita kasd gubergren, no sea takimata sanctus est Lorem ipsum dolor sit amet.

Lorem ipsum dolor sit amet, consetetur sadipscing elitr, sed diam nonumy eirmod tempor invidunt ut labore et dolore magna aliquyam erat, sed diam voluptua. At vero eos et accusam et justo duo dolores et ea rebum. Stet clita kasd gubergren, no sea takimata sanctus est Lorem ipsum dolor sit amet. Lorem ipsum dolor sit amet, consetetur sadipscing elitr, sed diam nonumy eirmod tempor invidunt ut labore et dolore magna aliquyam erat, sed diam voluptua. At vero eos et accusam et justo duo dolores et ea rebum. Stet clita kasd gubergren (vgl. Abbildung~\ref{fig:beispiel}), no sea takimata sanctus est Lorem ipsum dolor sit amet.

\begin{figure}[ht]
  \centering
  \begin{tikzpicture}
    \draw[draw=lightgray, step=1cm] (-5, -3) grid (5, 3);
    \draw[thick, -latex] (-4.5, 0) -- (4.5, 0) node[below] {$x$};
    \draw[thick, -latex] (0, -2.5) -- (0, 2.5) node[right] {$y$};
    \draw[draw=red, fill=red] (1.2, 0.3) circle (1cm);
    \draw[draw=blue, fill=blue] (-3.4, -1.7) rectangle (-1.5, 1.3);
  \end{tikzpicture}
  \caption{Ein Beispielbild, natürlich mit Bildunterschrift.}
  \label{fig:beispiel}
\end{figure}

Lorem ipsum dolor sit amet, consetetur sadipscing elitr, sed diam nonumy eirmod tempor invidunt ut labore et dolore magna aliquyam erat, sed diam voluptua. At vero eos et accusam et justo duo dolores et ea rebum. Stet clita kasd gubergren, no sea takimata sanctus est Lorem ipsum dolor sit amet. Lorem ipsum dolor sit amet, consetetur sadipscing elitr, sed diam nonumy eirmod tempor invidunt ut labore et dolore magna aliquyam erat, sed diam voluptua. At vero eos et accusam et justo duo dolores et ea rebum. Stet clita kasd gubergren, no sea takimata sanctus est Lorem ipsum dolor sit amet.

Lorem ipsum dolor sit amet, consetetur sadipscing elitr, sed diam nonumy eirmod tempor invidunt ut labore et dolore magna aliquyam erat, sed diam voluptua. At vero eos et accusam et justo duo dolores et ea rebum. Stet clita kasd gubergren, no sea takimata sanctus est Lorem ipsum dolor sit amet. Lorem ipsum dolor sit amet, consetetur sadipscing elitr, sed diam nonumy eirmod tempor invidunt ut labore et dolore magna aliquyam erat, sed diam voluptua. At vero eos et accusam et justo duo dolores et ea rebum. Stet clita kasd gubergren, no sea takimata sanctus est Lorem ipsum dolor sit amet.

Lorem ipsum dolor sit amet, consetetur sadipscing elitr, sed diam nonumy eirmod tempor invidunt ut labore et dolore magna aliquyam erat, sed diam voluptua. At vero eos et accusam et justo duo dolores et ea rebum. Stet clita kasd gubergren, no sea takimata sanctus est Lorem ipsum dolor sit amet. Lorem ipsum dolor sit amet, consetetur sadipscing elitr, sed diam nonumy eirmod tempor invidunt ut labore et dolore magna aliquyam erat, sed diam voluptua. At vero eos et accusam et justo duo dolores et ea rebum. Stet clita kasd gubergren, no sea takimata sanctus est Lorem ipsum dolor sit amet.
\section{Erster Abschnitt von Kapitel 1}
Lorem ipsum dolor sit amet, consetetur sadipscing elitr, sed diam nonumy eirmod tempor invidunt ut labore et dolore magna aliquyam erat, sed diam voluptua. At vero eos et accusam et justo duo dolores et ea rebum. Stet clita kasd gubergren, no sea takimata sanctus est Lorem ipsum dolor sit amet. Lorem ipsum dolor sit amet, consetetur sadipscing elitr, sed diam nonumy eirmod tempor invidunt ut labore et dolore magna aliquyam erat, sed diam voluptua. At vero eos et accusam et justo duo dolores et ea rebum. Stet clita kasd gubergren, no sea takimata sanctus est Lorem ipsum dolor sit amet.

Lorem ipsum dolor sit amet, consetetur sadipscing elitr, sed diam nonumy eirmod tempor invidunt ut labore et dolore magna aliquyam erat, sed diam voluptua. At vero eos et accusam et justo duo dolores et ea rebum. Stet clita kasd gubergren, no sea takimata sanctus est Lorem ipsum dolor sit amet. Lorem ipsum dolor sit amet, consetetur sadipscing elitr, sed diam nonumy eirmod tempor invidunt ut labore et dolore magna aliquyam erat, sed diam voluptua. At vero eos et accusam et justo duo dolores et ea rebum. Stet clita kasd gubergren, no sea takimata sanctus est Lorem ipsum dolor sit amet.

Lorem ipsum dolor sit amet, consetetur sadipscing elitr, sed diam nonumy eirmod tempor invidunt ut labore et dolore magna aliquyam erat, sed diam voluptua. At vero eos et accusam et justo duo dolores et ea rebum. Stet clita kasd gubergren, no sea takimata sanctus est Lorem ipsum dolor sit amet. Lorem ipsum dolor sit amet, consetetur sadipscing elitr, sed diam nonumy eirmod tempor invidunt ut labore et dolore magna aliquyam erat, sed diam voluptua. At vero eos et accusam et justo duo dolores et ea rebum. Stet clita kasd gubergren, no sea takimata sanctus est Lorem ipsum dolor sit amet.

Lorem ipsum dolor sit amet, consetetur sadipscing elitr, sed diam nonumy eirmod tempor invidunt ut labore et dolore magna aliquyam erat, sed diam voluptua. At vero eos et accusam et justo duo dolores et ea rebum. Stet clita kasd gubergren, no sea takimata sanctus est Lorem ipsum dolor sit amet. Lorem ipsum dolor sit amet, consetetur sadipscing elitr, sed diam nonumy eirmod tempor invidunt ut labore et dolore magna aliquyam erat, sed diam voluptua. At vero eos et accusam et justo duo dolores et ea rebum. Stet clita kasd gubergren, no sea takimata sanctus est Lorem ipsum dolor sit amet.

Lorem ipsum dolor sit amet, consetetur sadipscing elitr, sed diam nonumy eirmod tempor invidunt ut labore et dolore magna aliquyam erat, sed diam voluptua. At vero eos et accusam et justo duo dolores et ea rebum. Stet clita kasd gubergren, no sea takimata sanctus est Lorem ipsum dolor sit amet. Lorem ipsum dolor sit amet, consetetur sadipscing elitr, sed diam nonumy eirmod tempor invidunt ut labore et dolore magna aliquyam erat, sed diam voluptua. At vero eos et accusam et justo duo dolores et ea rebum. Stet clita kasd gubergren, no sea takimata sanctus est Lorem ipsum dolor sit amet.

Lorem ipsum dolor sit amet, consetetur sadipscing elitr, sed diam nonumy eirmod tempor invidunt ut labore et dolore magna aliquyam erat, sed diam voluptua. At vero eos et accusam et justo duo dolores et ea rebum. Stet clita kasd gubergren, no sea takimata sanctus est Lorem ipsum dolor sit amet. Lorem ipsum dolor sit amet, consetetur sadipscing elitr, sed diam nonumy eirmod tempor invidunt ut labore et dolore magna aliquyam erat, sed diam voluptua. At vero eos et accusam et justo duo dolores et ea rebum. Stet clita kasd gubergren, no sea takimata sanctus est Lorem ipsum dolor sit amet.
\section{Zweiter Abschnitt von Kapitel 1}
Lorem ipsum dolor sit amet, consetetur sadipscing elitr, sed diam nonumy eirmod tempor invidunt ut labore et dolore magna aliquyam erat, sed diam voluptua. At vero eos et accusam et justo duo dolores et ea rebum. Stet clita kasd gubergren, no sea takimata sanctus est Lorem ipsum dolor sit amet. Lorem ipsum dolor sit amet, consetetur sadipscing elitr, sed diam nonumy eirmod tempor invidunt ut labore et dolore magna aliquyam erat, sed diam voluptua. At vero eos et accusam et justo duo dolores et ea rebum. Stet clita kasd gubergren, no sea takimata sanctus est Lorem ipsum dolor sit amet.

Lorem ipsum dolor sit amet, consetetur sadipscing elitr, sed diam nonumy eirmod tempor invidunt ut labore et dolore magna aliquyam erat, sed diam voluptua. At vero eos et accusam et justo duo dolores et ea rebum. Stet clita kasd gubergren, no sea takimata sanctus est Lorem ipsum dolor sit amet. Lorem ipsum dolor sit amet, consetetur sadipscing elitr, sed diam nonumy eirmod tempor invidunt ut labore et dolore magna aliquyam erat, sed diam voluptua. At vero eos et accusam et justo duo dolores et ea rebum. Stet clita kasd gubergren, no sea takimata sanctus est Lorem ipsum dolor sit amet.

Lorem ipsum dolor sit amet, consetetur sadipscing elitr, sed diam nonumy eirmod tempor invidunt ut labore et dolore magna aliquyam erat, sed diam voluptua. At vero eos et accusam et justo duo dolores et ea rebum. Stet clita kasd gubergren, no sea takimata sanctus est Lorem ipsum dolor sit amet. Lorem ipsum dolor sit amet, consetetur sadipscing elitr, sed diam nonumy eirmod tempor invidunt ut labore et dolore magna aliquyam erat, sed diam voluptua. At vero eos et accusam et justo duo dolores et ea rebum. Stet clita kasd gubergren, no sea takimata sanctus est Lorem ipsum dolor sit amet.

Lorem ipsum dolor sit amet, consetetur sadipscing elitr, sed diam nonumy eirmod tempor invidunt ut labore et dolore magna aliquyam erat, sed diam voluptua. At vero eos et accusam et justo duo dolores et ea rebum. Stet clita kasd gubergren, no sea takimata sanctus est Lorem ipsum dolor sit amet. Lorem ipsum dolor sit amet, consetetur sadipscing elitr, sed diam nonumy eirmod tempor invidunt ut labore et dolore magna aliquyam erat, sed diam voluptua. At vero eos et accusam et justo duo dolores et ea rebum. Stet clita kasd gubergren, no sea takimata sanctus est Lorem ipsum dolor sit amet.

Lorem ipsum dolor sit amet, consetetur sadipscing elitr, sed diam nonumy eirmod tempor invidunt ut labore et dolore magna aliquyam erat, sed diam voluptua. At vero eos et accusam et justo duo dolores et ea rebum. Stet clita kasd gubergren, no sea takimata sanctus est Lorem ipsum dolor sit amet. Lorem ipsum dolor sit amet, consetetur sadipscing elitr, sed diam nonumy eirmod tempor invidunt ut labore et dolore magna aliquyam erat, sed diam voluptua. At vero eos et accusam et justo duo dolores et ea rebum. Stet clita kasd gubergren, no sea takimata sanctus est Lorem ipsum dolor sit amet.

Lorem ipsum dolor sit amet, consetetur sadipscing elitr, sed diam nonumy eirmod tempor invidunt ut labore et dolore magna aliquyam erat, sed diam voluptua. At vero eos et accusam et justo duo dolores et ea rebum. Stet clita kasd gubergren, no sea takimata sanctus est Lorem ipsum dolor sit amet. Lorem ipsum dolor sit amet, consetetur sadipscing elitr, sed diam nonumy eirmod tempor invidunt ut labore et dolore magna aliquyam erat, sed diam voluptua. At vero eos et accusam et justo duo dolores et ea rebum. Stet clita kasd gubergren, no sea takimata sanctus est Lorem ipsum dolor sit amet.
\subsubsection{Ein Unterabschnitt}
Lorem ipsum dolor sit amet, consetetur sadipscing elitr, sed diam nonumy eirmod tempor invidunt ut labore et dolore magna aliquyam erat, sed diam voluptua. At vero eos et accusam et justo duo dolores et ea rebum. Stet clita kasd gubergren, no sea takimata sanctus est Lorem ipsum dolor sit amet. Lorem ipsum dolor sit amet, consetetur sadipscing elitr, sed diam nonumy eirmod tempor invidunt ut labore et dolore magna aliquyam erat, sed diam voluptua. At vero eos et accusam et justo duo dolores et ea rebum. Stet clita kasd gubergren, no sea takimata sanctus est Lorem ipsum dolor sit amet.

Lorem ipsum dolor sit amet, consetetur sadipscing elitr, sed diam nonumy eirmod tempor invidunt ut labore et dolore magna aliquyam erat, sed diam voluptua. At vero eos et accusam et justo duo dolores et ea rebum. Stet clita kasd gubergren, no sea takimata sanctus est Lorem ipsum dolor sit amet. Lorem ipsum dolor sit amet, consetetur sadipscing elitr, sed diam nonumy eirmod tempor invidunt ut labore et dolore magna aliquyam erat, sed diam voluptua. At vero eos et accusam et justo duo dolores et ea rebum. Stet clita kasd gubergren, no sea takimata sanctus est Lorem ipsum dolor sit amet.

Lorem ipsum dolor sit amet, consetetur sadipscing elitr, sed diam nonumy eirmod tempor invidunt ut labore et dolore magna aliquyam erat, sed diam voluptua. At vero eos et accusam et justo duo dolores et ea rebum. Stet clita kasd gubergren, no sea takimata sanctus est Lorem ipsum dolor sit amet. Lorem ipsum dolor sit amet, consetetur sadipscing elitr, sed diam nonumy eirmod tempor invidunt ut labore et dolore magna aliquyam erat, sed diam voluptua. At vero eos et accusam et justo duo dolores et ea rebum. Stet clita kasd gubergren, no sea takimata sanctus est Lorem ipsum dolor sit amet.

Lorem ipsum dolor sit amet, consetetur sadipscing elitr, sed diam nonumy eirmod tempor invidunt ut labore et dolore magna aliquyam erat, sed diam voluptua. At vero eos et accusam et justo duo dolores et ea rebum. Stet clita kasd gubergren, no sea takimata sanctus est Lorem ipsum dolor sit amet. Lorem ipsum dolor sit amet, consetetur sadipscing elitr, sed diam nonumy eirmod tempor invidunt ut labore et dolore magna aliquyam erat, sed diam voluptua. At vero eos et accusam et justo duo dolores et ea rebum. Stet clita kasd gubergren, no sea takimata sanctus est Lorem ipsum dolor sit amet.
  \chapter{Vorgehensweise}
\label{chap:2}
  \chapter{Implementierung der Karte}
Lorem ipsum dolor sit amet, consetetur sadipscing elitr, sed diam nonumy eirmod tempor invidunt ut labore et dolore magna aliquyam erat, sed diam voluptua. At vero eos et accusam et justo duo dolores et ea rebum. Stet clita kasd gubergren, no sea takimata sanctus est Lorem ipsum dolor sit amet. Lorem ipsum dolor sit amet, consetetur sadipscing elitr, sed diam nonumy eirmod tempor invidunt ut labore et dolore magna aliquyam erat, sed diam voluptua. At vero eos et accusam et justo duo dolores et ea rebum. Stet clita kasd gubergren, no sea takimata sanctus est Lorem ipsum dolor sit amet.

Lorem ipsum dolor sit amet, consetetur sadipscing elitr, sed diam nonumy eirmod tempor invidunt ut labore et dolore magna aliquyam erat, sed diam voluptua. At vero eos et accusam et justo duo dolores et ea rebum. Stet clita kasd gubergren, no sea takimata sanctus est Lorem ipsum dolor sit amet. Lorem ipsum dolor sit amet, consetetur sadipscing elitr, sed diam nonumy eirmod tempor invidunt ut labore et dolore magna aliquyam erat, sed diam voluptua. At vero eos et accusam et justo duo dolores et ea rebum. Stet clita kasd gubergren, no sea takimata sanctus est Lorem ipsum dolor sit amet.

\begin{align}
  \sin(x+2\pi) &= \sin(x)\\
  \cos(x+2\pi) &= \cos(x)\\
  \cos(x) &= \sin\left(x+\frac{\pi}{2}\right)
\end{align}

Lorem ipsum dolor sit amet, consetetur sadipscing elitr, sed diam nonumy eirmod tempor invidunt ut labore et dolore magna aliquyam erat, sed diam voluptua. At vero eos et accusam et justo duo dolores et ea rebum. Stet clita kasd gubergren, no sea takimata sanctus est Lorem ipsum dolor sit amet. Lorem ipsum dolor sit amet, consetetur sadipscing elitr, sed diam nonumy eirmod tempor invidunt ut labore et dolore magna aliquyam erat, sed diam voluptua. At vero eos et accusam et justo duo dolores et ea rebum. Stet clita kasd gubergren, no sea takimata sanctus est Lorem ipsum dolor sit amet.

Lorem ipsum dolor sit amet, consetetur sadipscing elitr, sed diam nonumy eirmod tempor invidunt ut labore et dolore magna aliquyam erat, sed diam voluptua. At vero eos et accusam et justo duo dolores et ea rebum. Stet clita kasd gubergren, no sea takimata sanctus est Lorem ipsum dolor sit amet. Lorem ipsum dolor sit amet, consetetur sadipscing elitr, sed diam nonumy eirmod tempor invidunt ut labore et dolore magna aliquyam erat, sed diam voluptua. At vero eos et accusam et justo duo dolores et ea rebum. Stet clita kasd gubergren, no sea takimata sanctus est Lorem ipsum dolor sit amet.

Lorem ipsum dolor sit amet, consetetur sadipscing elitr, sed diam nonumy eirmod tempor invidunt ut labore et dolore magna aliquyam erat, sed diam voluptua. At vero eos et accusam et justo duo dolores et ea rebum. Stet clita kasd gubergren, no sea takimata sanctus est Lorem ipsum dolor sit amet. Lorem ipsum dolor sit amet, consetetur sadipscing elitr, sed diam nonumy eirmod tempor invidunt ut labore et dolore magna aliquyam erat, sed diam voluptua. At vero eos et accusam et justo duo dolores et ea rebum. Stet clita kasd gubergren, no sea takimata sanctus est Lorem ipsum dolor sit amet.

Lorem ipsum dolor sit amet, consetetur sadipscing elitr, sed diam nonumy eirmod tempor invidunt ut labore et dolore magna aliquyam erat, sed diam voluptua. At vero eos et accusam et justo duo dolores et ea rebum. Stet clita kasd gubergren, no sea takimata sanctus est Lorem ipsum dolor sit amet. Lorem ipsum dolor sit amet, consetetur sadipscing elitr, sed diam nonumy eirmod tempor invidunt ut labore et dolore magna aliquyam erat, sed diam voluptua. At vero eos et accusam et justo duo dolores et ea rebum. Stet clita kasd gubergren, no sea takimata sanctus est Lorem ipsum dolor sit amet.
\section{Erster Abschnitt von Kapitel 2}
Lorem ipsum dolor sit amet, consetetur sadipscing elitr, sed diam nonumy eirmod tempor invidunt ut labore et dolore magna aliquyam erat, sed diam voluptua. At vero eos et accusam et justo duo dolores et ea rebum. Stet clita kasd gubergren, no sea takimata sanctus est Lorem ipsum dolor sit amet. Lorem ipsum dolor sit amet, consetetur sadipscing elitr, sed diam nonumy eirmod tempor invidunt ut labore et dolore magna aliquyam erat, sed diam voluptua. At vero eos et accusam et justo duo dolores et ea rebum (vgl. Tabelle~\ref{tab:beispiel}). Stet clita kasd gubergren, no sea takimata sanctus est Lorem ipsum dolor sit amet.

\begin{table}
  \centering
  \begin{tabular}{l|c|r}
	links & mittig & rechts\\\hline
	a & b & c\\
	1 & 2 & 3  
  \end{tabular}
  \caption{Eine Beispieltabelle, natürlich mit einer Erläuterung.}
  \label{tab:beispiel}
\end{table}

Lorem ipsum dolor sit amet, consetetur sadipscing elitr, sed diam nonumy eirmod tempor invidunt ut labore et dolore magna aliquyam erat, sed diam voluptua. At vero eos et accusam et justo duo dolores et ea rebum. Stet clita kasd gubergren, no sea takimata sanctus est Lorem ipsum dolor sit amet. Lorem ipsum dolor sit amet, consetetur sadipscing elitr, sed diam nonumy eirmod tempor invidunt ut labore et dolore magna aliquyam erat, sed diam voluptua. At vero eos et accusam et justo duo dolores et ea rebum. Stet clita kasd gubergren, no sea takimata sanctus est Lorem ipsum dolor sit amet.

Lorem ipsum dolor sit amet, consetetur sadipscing elitr, sed diam nonumy eirmod tempor invidunt ut labore et dolore magna aliquyam erat, sed diam voluptua. At vero eos et accusam et justo duo dolores et ea rebum. Stet clita kasd gubergren, no sea takimata sanctus est Lorem ipsum dolor sit amet. Lorem ipsum dolor sit amet, consetetur sadipscing elitr, sed diam nonumy eirmod tempor invidunt ut labore et dolore magna aliquyam erat, sed diam voluptua. At vero eos et accusam et justo duo dolores et ea rebum. Stet clita kasd gubergren, no sea takimata sanctus est Lorem ipsum dolor sit amet.

Lorem ipsum dolor sit amet, consetetur sadipscing elitr, sed diam nonumy eirmod tempor invidunt ut labore et dolore magna aliquyam erat, sed diam voluptua. At vero eos et accusam et justo duo dolores et ea rebum. Stet clita kasd gubergren, no sea takimata sanctus est Lorem ipsum dolor sit amet. Lorem ipsum dolor sit amet, consetetur sadipscing elitr, sed diam nonumy eirmod tempor invidunt ut labore et dolore magna aliquyam erat, sed diam voluptua. At vero eos et accusam et justo duo dolores et ea rebum. Stet clita kasd gubergren, no sea takimata sanctus est Lorem ipsum dolor sit amet.

Lorem ipsum dolor sit amet, consetetur sadipscing elitr, sed diam nonumy eirmod tempor invidunt ut labore et dolore magna aliquyam erat, sed diam voluptua. At vero eos et accusam et justo duo dolores et ea rebum. Stet clita kasd gubergren, no sea takimata sanctus est Lorem ipsum dolor sit amet. Lorem ipsum dolor sit amet, consetetur sadipscing elitr, sed diam nonumy eirmod tempor invidunt ut labore et dolore magna aliquyam erat, sed diam voluptua. At vero eos et accusam et justo duo dolores et ea rebum. Stet clita kasd gubergren, no sea takimata sanctus est Lorem ipsum dolor sit amet.

Lorem ipsum dolor sit amet, consetetur sadipscing elitr, sed diam nonumy eirmod tempor invidunt ut labore et dolore magna aliquyam erat, sed diam voluptua. At vero eos et accusam et justo duo dolores et ea rebum. Stet clita kasd gubergren, no sea takimata sanctus est Lorem ipsum dolor sit amet. Lorem ipsum dolor sit amet, consetetur sadipscing elitr, sed diam nonumy eirmod tempor invidunt ut labore et dolore magna aliquyam erat, sed diam voluptua. At vero eos et accusam et justo duo dolores et ea rebum. Stet clita kasd gubergren, no sea takimata sanctus est Lorem ipsum dolor sit amet.

Lorem ipsum dolor sit amet, consetetur sadipscing elitr, sed diam nonumy eirmod tempor invidunt ut labore et dolore magna aliquyam erat, sed diam voluptua. At vero eos et accusam et justo duo dolores et ea rebum. Stet clita kasd gubergren, no sea takimata sanctus est Lorem ipsum dolor sit amet. Lorem ipsum dolor sit amet, consetetur sadipscing elitr, sed diam nonumy eirmod tempor invidunt ut labore et dolore magna aliquyam erat, sed diam voluptua. At vero eos et accusam et justo duo dolores et ea rebum. Stet clita kasd gubergren, no sea takimata sanctus est Lorem ipsum dolor sit amet.

Lorem ipsum dolor sit amet, consetetur sadipscing elitr, sed diam nonumy eirmod tempor invidunt ut labore et dolore magna aliquyam erat, sed diam voluptua. At vero eos et accusam et justo duo dolores et ea rebum. Stet clita kasd gubergren, no sea takimata sanctus est Lorem ipsum dolor sit amet. Lorem ipsum dolor sit amet, consetetur sadipscing elitr, sed diam nonumy eirmod tempor invidunt ut labore et dolore magna aliquyam erat, sed diam voluptua. At vero eos et accusam et justo duo dolores et ea rebum. Stet clita kasd gubergren, no sea takimata sanctus est Lorem ipsum dolor sit amet.
\section{Zweiter Abschnitt von Kapitel 2}
Lorem ipsum dolor sit amet, consetetur sadipscing elitr, sed diam nonumy eirmod tempor invidunt ut labore et dolore magna aliquyam erat, sed diam voluptua. At vero eos et accusam et justo duo dolores et ea rebum. Stet clita kasd gubergren, no sea takimata sanctus est Lorem ipsum dolor sit amet. Lorem ipsum dolor sit amet, consetetur sadipscing elitr, sed diam nonumy eirmod tempor invidunt ut labore et dolore magna aliquyam erat, sed diam voluptua. At vero eos et accusam et justo duo dolores et ea rebum. Stet clita kasd gubergren, no sea takimata sanctus est Lorem ipsum dolor sit amet.

Lorem ipsum dolor sit amet, consetetur sadipscing elitr, sed diam nonumy eirmod tempor invidunt ut labore et dolore magna aliquyam erat, sed diam voluptua. At vero eos et accusam et justo duo dolores et ea rebum. Stet clita kasd gubergren, no sea takimata sanctus est Lorem ipsum dolor sit amet. Lorem ipsum dolor sit amet, consetetur sadipscing elitr, sed diam nonumy eirmod tempor invidunt ut labore et dolore magna aliquyam erat, sed diam voluptua. At vero eos et accusam et justo duo dolores et ea rebum. Stet clita kasd gubergren, no sea takimata sanctus est Lorem ipsum dolor sit amet.

Lorem ipsum dolor sit amet, consetetur sadipscing elitr, sed diam nonumy eirmod tempor invidunt ut labore et dolore magna aliquyam erat, sed diam voluptua. At vero eos et accusam et justo duo dolores et ea rebum. Stet clita kasd gubergren, no sea takimata sanctus est Lorem ipsum dolor sit amet. Lorem ipsum dolor sit amet, consetetur sadipscing elitr, sed diam nonumy eirmod tempor invidunt ut labore et dolore magna aliquyam erat, sed diam voluptua. At vero eos et accusam et justo duo dolores et ea rebum. Stet clita kasd gubergren, no sea takimata sanctus est Lorem ipsum dolor sit amet.

Lorem ipsum dolor sit amet, consetetur sadipscing elitr, sed diam nonumy eirmod tempor invidunt ut labore et dolore magna aliquyam erat, sed diam voluptua. At vero eos et accusam et justo duo dolores et ea rebum. Stet clita kasd gubergren, no sea takimata sanctus est Lorem ipsum dolor sit amet. Lorem ipsum dolor sit amet, consetetur sadipscing elitr, sed diam nonumy eirmod tempor invidunt ut labore et dolore magna aliquyam erat, sed diam voluptua. At vero eos et accusam et justo duo dolores et ea rebum. Stet clita kasd gubergren, no sea takimata sanctus est Lorem ipsum dolor sit amet.

Lorem ipsum dolor sit amet, consetetur sadipscing elitr, sed diam nonumy eirmod tempor invidunt ut labore et dolore magna aliquyam erat, sed diam voluptua. At vero eos et accusam et justo duo dolores et ea rebum. Stet clita kasd gubergren, no sea takimata sanctus est Lorem ipsum dolor sit amet. Lorem ipsum dolor sit amet, consetetur sadipscing elitr, sed diam nonumy eirmod tempor invidunt ut labore et dolore magna aliquyam erat, sed diam voluptua. At vero eos et accusam et justo duo dolores et ea rebum. Stet clita kasd gubergren, no sea takimata sanctus est Lorem ipsum dolor sit amet.

Lorem ipsum dolor sit amet, consetetur sadipscing elitr, sed diam nonumy eirmod tempor invidunt ut labore et dolore magna aliquyam erat, sed diam voluptua. At vero eos et accusam et justo duo dolores et ea rebum. Stet clita kasd gubergren, no sea takimata sanctus est Lorem ipsum dolor sit amet. Lorem ipsum dolor sit amet, consetetur sadipscing elitr, sed diam nonumy eirmod tempor invidunt ut labore et dolore magna aliquyam erat, sed diam voluptua. At vero eos et accusam et justo duo dolores et ea rebum. Stet clita kasd gubergren, no sea takimata sanctus est Lorem ipsum dolor sit amet.

Lorem ipsum dolor sit amet, consetetur sadipscing elitr, sed diam nonumy eirmod tempor invidunt ut labore et dolore magna aliquyam erat, sed diam voluptua. At vero eos et accusam et justo duo dolores et ea rebum. Stet clita kasd gubergren, no sea takimata sanctus est Lorem ipsum dolor sit amet. Lorem ipsum dolor sit amet, consetetur sadipscing elitr, sed diam nonumy eirmod tempor invidunt ut labore et dolore magna aliquyam erat, sed diam voluptua. At vero eos et accusam et justo duo dolores et ea rebum. Stet clita kasd gubergren, no sea takimata sanctus est Lorem ipsum dolor sit amet.

Lorem ipsum dolor sit amet, consetetur sadipscing elitr, sed diam nonumy eirmod tempor invidunt ut labore et dolore magna aliquyam erat, sed diam voluptua. At vero eos et accusam et justo duo dolores et ea rebum. Stet clita kasd gubergren, no sea takimata sanctus est Lorem ipsum dolor sit amet. Lorem ipsum dolor sit amet, consetetur sadipscing elitr, sed diam nonumy eirmod tempor invidunt ut labore et dolore magna aliquyam erat, sed diam voluptua. At vero eos et accusam et justo duo dolores et ea rebum. Stet clita kasd gubergren, no sea takimata sanctus est Lorem ipsum dolor sit amet.

Lorem ipsum dolor sit amet, consetetur sadipscing elitr, sed diam nonumy eirmod tempor invidunt ut labore et dolore magna aliquyam erat, sed diam voluptua. At vero eos et accusam et justo duo dolores et ea rebum. Stet clita kasd gubergren, no sea takimata sanctus est Lorem ipsum dolor sit amet. Lorem ipsum dolor sit amet, consetetur sadipscing elitr, sed diam nonumy eirmod tempor invidunt ut labore et dolore magna aliquyam erat, sed diam voluptua. At vero eos et accusam et justo duo dolores et ea rebum. Stet clita kasd gubergren, no sea takimata sanctus est Lorem ipsum dolor sit amet.

Lorem ipsum dolor sit amet, consetetur sadipscing elitr, sed diam nonumy eirmod tempor invidunt ut labore et dolore magna aliquyam erat, sed diam voluptua. At vero eos et accusam et justo duo dolores et ea rebum. Stet clita kasd gubergren, no sea takimata sanctus est Lorem ipsum dolor sit amet. Lorem ipsum dolor sit amet, consetetur sadipscing elitr, sed diam nonumy eirmod tempor invidunt ut labore et dolore magna aliquyam erat, sed diam voluptua. At vero eos et accusam et justo duo dolores et ea rebum. Stet clita kasd gubergren, no sea takimata sanctus est Lorem ipsum dolor sit amet.
  \chapter{Implementierung der Liste}
\label{chap:4}

\begin{wrapfigure}{R}{0.5\textwidth}
	\vspace{-\baselineskip}
	\centering
	\includegraphics[scale=0.15]{img/list}
	\caption{Ansicht des Listen-Tabs mit der alphabetisch geordneten Liste an Parkhäusern}
	\label{fig:list}
\end{wrapfigure}
Alle Komponenten, aus denen die Liste besteht, sind im Ordner \verb|ParkingList| enthalten. Für die Liste selbst wird die FlatList-Komponente, welche standardmäßig im react-native Framework enthalten ist, verwendet. Dafür wird durch eine Funktion ein Array aus den dynamischen und statischen Daten erzeugt, welche an die FlatList-Komponente zur Anzeige übergeben wird. Die Erzeugung und Anzeige des Arrays ist in der Datei \verb|ParkingList.tsx| zu finden. Die anzuzeigenden Daten sind zum einen der Name des Parkhauses zur Identifizierung. Darunter wird dargestellt, ob das Parkhaus geöffnet oder geschlossen ist. Wenn das Parkhaus offen ist, wird dies in grüner Schrift angezeigt, geschlossen wird dagegen rot geschrieben. Da alle Parkhäuser 24 Stunden offen sind, sollte ein geschlossenes Parkhaus nur bei Wartungsarbeiten und Störungen auftreten. Rechts daneben wird die Entfernung des Nutzers zu diesem spezifischen Parkhaus in Metern angezeigt. Die Position des Nutzers wird wieder über dieselbe Methode erlangt, wie in \autoref{chap:3} für das Geofencing. Über die Haversine Distanz wird wieder die Entfernung zu den Koordinaten der Parkhäuser berechnet und dann angezeigt. Dies wird bei jeder Positionsänderung des Nutzers durchgeführt, damit die Entfernungen immer aktuell sind. Darunter wird noch der Trend des Parkhauses über Icons symbolisiert. Ein grüner Pfeil nach unten signalisiert, dass sich das Parkhaus leert, ein roter Pfeil nach über bedeutet ein sich füllendes Parkhaus und ein schwarzer Balken zeigt einen gleichbleibenden Trend an, wie mit den Markern der Karte.

\begin{wrapfigure}{R}{0.5\textwidth}
	\vspace{-\baselineskip}
	\centering
	\includegraphics[scale=0.15]{img/sorting}
	\caption{Aufruf des Menüs zum Sortieren der Liste}
	\label{fig:sorting}
\end{wrapfigure}
Diese Liste kann zudem auch konfiguriert werden. Dafür ist zu einen links oben der Liste ein Schalter vorhanden, für den die Switch-Komponente von react-native benutzt wurde, mit der Beschreibung ,,Nur Favoriten''. Wenn dieser Schalter gesetzt ist, werden nur die Parkhäuser angezeigt, welche als Favorit markiert wurden, also bei welchen das rote Herz ausgefüllt ist, wie in \autoref{fig:detail_map} zu sehen ist. Zudem ist rechts oben ein Sortier-Icon zu sehen. Wenn auf dieses Icon geklickt wird öffnet sich eine Liste an Sortiermöglichkeiten, wobei dieses Menü wieder aus dem Paket react-native-paper stammt. Diese Liste ist in \autoref{fig:sorting} ersichtlich. Es gibt drei Sortiermöglichkeiten. Die erste sortiert alphabetisch nach den Namen der Parkhäuser. Die zweite sortiert aufsteigend nach den Entfernungen, sodass das näheste Parkhaus ganz oben steht. Da die Entfernungen bei jeder Positionsänderung des Nutzers neu berechnet werden, wird auch die Liste neu sortiert, wenn ein Parkhaus nach einer Positionsänderung näher als das andere am Nutzer ist. Somit ist das näheste Parkhaus immer oben. Die letzte Möglichkeit sortiert absteigend nach den freien Plätzen, also je mehr Parkplätze in einem Parkhaus frei sind, desto weiter oben ist es in der Liste.

Da es nicht möglich war, alle Daten zu den Parkhäusern in der Liste anzuzeigen, mussten Detail-Fenster erzeugt werden, die nähere Informationen zu den Parkhäusern darstellen. Diese Detail-Fenster sind über Anklicken des zugehörigen Listeneintrags aufrufbar. Um zwischen der Liste und den verschiedenen Detail-Fenstern der Parkhäuser navigieren zu können, wurde das Paket React Navigation benutzt. Dieses ermöglicht das Aufrufen von Fenstern über die Screen-Komponente. Diese Fenster besitzen einen Zurück-Knopf, wie in \autoref{fig:details1} ersichtlich, mit dem zum letzten Fenster, also der Liste, zurück navigiert werden kann. Das Detail-Fenster besitzt zudem auf die Navigations- und Favoriten-Icons der Karte, welche hier dieselbe Funktion besitzen. Durch Drücken des Navigations-Icons wird der Nutzer hier jedoch entweder zur Karte für die interne Navigation, oder direkt zu Google Maps weitergeleitet.
\newpage

\begin{wrapfigure}{r}{0.5\textwidth}
	\vspace{-\baselineskip}
	\centering
	\includegraphics[scale=0.13]{img/details1}
	\caption{Erster Teil des Detail-Fensters zum Parkhaus am Ziegeltor}
	\label{fig:details1}
\end{wrapfigure}
Zur Anzeige der restlichen Informationen des Parkhauses wurde wieder das react-native-paper Paket benutzt. Hier bestehen die weißen Kasten, aus denen sich das Detail-Fenster zusammensetzt, aus der Card Komponente. Zuerst wird durch rote oder grüne Schrift gezeigt, ob es in diesem Parkhaus gerade eine Störung gibt oder nicht und ob das Parkhaus offen ist. Danach werden die Öffnungszeiten angeschrieben und der Trend nochmals schriftlich gezeigt. Darunter wird die Anzahl der freien Parkplätze auf dieselbe Weise wie in der Karte gezeigt.

\begin{wrapfigure}{r}{0.5\textwidth}
	\vspace{-\baselineskip}
	\centering
	\includegraphics[scale=0.12]{img/details2}
	\caption{Zweiter Teil des Detail-Fensters zum Parkhaus am Ziegeltor}
	\label{fig:details2}
\end{wrapfigure}
Unter den freien Parkplätzen sind die Preise dargestellt. Wenn eine Unterscheidung in Tag- und Nachtpreise nötig ist, wie beim Parkhaus am Ziegeltor, werden diese Preise als Tabelle gezeigt, wobei hier jeweils der Preis für die erste und die weiteren Stunden eingetragen ist. Die Tabelle besteht aus der DataTable-Komponente des react-native-paper Pakets. Unter den Preisen sind nur noch die Zusatzinformationen der statischen Daten zu sehen. Damit werden alle verfügbaren Daten angezeigt und dem Nutzer übersichtlich und benutzerfreundlich dargestellt.

Damit sind alle Komponenten und Funktionalitäten der Liste beschrieben worden. Um die Funktionsweise und den Effekt der Navigationsknöpfe in der Karte und in den Detail-Fenstern der Parkhäuser besser verstehen zu können, wird im nächsten Kapitel auf die interne Pseudo-Navigation eingegangen, welche für diese App erstellt wurde und wie diese Navigation reagiert, wenn kein Weg zum Parkhaus gefunden werden kann.


  \chapter{Umsetzung einer Pseudo-Navigation}
\label{chap:5}
  \chapter{Zusätzliche Funktionen der App}
\label{chap:6}

\begin{itemize}
	\item Tab-Navigation zwischen Liste und Karte
	\item Navigations-Knopf unten rechts mit Auswahl, ob nächstes Parkhaus oder Parkhaus in Geofence
	\item Genaue Preisdaten als Tabelle in Details zu Parkhaus
	\item Navigation von überall in der App zu jedem Parkhaus möglich
\end{itemize}
  \chapter{Ausblick und Fazit}
\label{chap:7}

\begin{itemize}
	\item Pseudo-Navigation verbessern, vielleicht auch durch Directions API von Google => kostet Geld
	\item Mehr Einstellungsmöglichkeiten (Darkmode)
	\item Mehr GPX-Dateien für bessere Abdeckung und weniger große Abstände zwischen Nutzer und Startpunkt der Navigation
	\item Genauere Berechnung des nächsten Parkhauses => nicht über Luftlinie sondern über Weg => Directions API nötig
	\item Splash Screen und Icon der App
\end{itemize}
  % ... weitere Kapitel
 
  % Literaturverzeichnis
  \phantomsection
  \addcontentsline{toc}{chapter}{Literaturverzeichnis}
  \begin{thebibliography}{10}
    \bibitem{ReactNative} [1] „React Native · Learn once, write anywhere“. \url{https://reactnative.dev/} (zugegriffen 11. Januar 2023).
    \bibitem{expo} „Expo“. \url{https://expo.dev/} (zugegriffen 13. Januar 2023).
    \bibitem{expo-sqlite} „SQLite“, Expo Documentation. \url{https://docs.expo.dev/versions/latest/sdk/sqlite} (zugegriffen 11. Dezember 2022).
    \bibitem{async-storage} „React Native Async Storage“. AsyncStorage, 13. Januar 2023. Zugegriffen: 13. Januar 2023. [Online]. Verfügbar unter: \url{https://github.com/react-native-async-storage/async-storage}
    \bibitem{maps} „react-native-maps“. react-native-maps, 13. Januar 2023. Zugegriffen: 13. Januar 2023. [Online]. Verfügbar unter: \url{https://github.com/react-native-maps/react-native-maps}
    \bibitem{location} „Location - Expo Documentation“. \url{https://docs.expo.dev/versions/latest/sdk/location/} (zugegriffen 11. Dezember 2022).
    \bibitem{progress} J. Arvidsson, „react-native-progress“. 6. Januar 2023. Zugegriffen: 8. Januar 2023. [Online]. Verfügbar unter: \url{https://github.com/oblador/react-native-progress}
    \bibitem{vector-icons} J. Arvidsson, „Multi-style fonts“. 7. Januar 2023. Zugegriffen: 7. Januar 2023. [Online]. Verfügbar unter: \url{https://github.com/oblador/react-native-vector-icons}
    \bibitem{haversine} „Distance on a sphere: The Haversine Formula“, Esri Community, 5. Oktober 2017. \url{https://community.esri.com/t5/coordinate-reference-systems-blog/distance-on-a-sphere-the-haversine-formula/ba-p/902128} (zugegriffen 7. Januar 2023).
    \bibitem{tts} „Speech“, Expo Documentation. \url{https://docs.expo.dev/versions/latest/sdk/speech} (zugegriffen 13. Januar 2023).
    
    
    
    
    
    
    
    
  \end{thebibliography}
  \newpage
  
  % Anhang
  \phantomsection
  \addcontentsline{toc}{chapter}{Abbildungsverzeichnis}
  \listoffigures
  \newpage
  
    \phantomsection
  \addcontentsline{toc}{chapter}{Listings}
  \lstlistoflistings
  \newpage

%  \phantomsection
%  \addcontentsline{toc}{chapter}{Tabellenverzeichnis}
%  \listoftables
%  \newpage
  
  %\include{anhang}
\end{document}    